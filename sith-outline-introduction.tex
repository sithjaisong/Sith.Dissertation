%=========================
%INTRODUCTION
%=========================
%\section*{Introduction}
\addcontentsline{toc}{chapter}{Introduction}
% 1. Threats do not reduce yields. Pests and diseases do. Rewrite.
% 2. Cite Oerke for actual and potential losses to bolster your case.
% 3. Your second sentence is out of place with the opening. You need to work on making the paragraph flow.
% 4. I'd argue that it's important for not just rice consuming societies but globally due to the interconnectedness of our modern societies.
The threats of pests and diseases to global rice production are significant yield reducing factors. Additionally, future rice production will need to grow by 2.4\% per year in order to meet the demands of a growing population \shortcite{Ray:2013by}. Addressing these yield-reducing factors is essential for food security in rice consuming societies now and in the future. 

% first sentence is incomprehensible
% the second sentence is an incomplete thought, untie what?
% grammar in the third sentence
% fourth sentence, if they have not been insufficient then why would you do this? Obviously they succeeded
% season is a noun, not an adjective
Description of pest management is complicated because human activities regarding with agricultural practices to the pests are diverse. To achieve this, the knowledge of relationships between pests and human activities including environmental factors are needed to untie \shortcite{Savary:2000ws}. While these studies are important in characterization the of pattern of pest injuries and the pattern of components relating to production situation, and how it related to each others. They have not been insufficient for explanation about change of relationships of injuries and components of production situation in term of time and regions, or how such relationships may change with climate variations.  For instance, the  relations between insect pest and weed were found only in the certain time, or certain locations. Over the long run, in season correlation between the frequency of patterns of pest incidence at different locations are also important to design the pest management strategies.

% this paragraph seems out of place with the first two. You need to work on flow. Your opening sentence should really be something more like your second sentence here.
We live in an increasingly connected era where by early 2013 over 90\% of all data had been generated to that point \shortcite{SINTEF}. In agriculture we should strive to find ways to harness our own ''big data''. We have a daunting task ahead of us that will require a myriad of approaches from well-know and understood to new approaches like big data analytics. Using new approaches to analyze so called ``big data'', we can start making new discoveries in relationships between factors of which we were previously unaware. These newly discovered relationships could be useful in designing and developing methods for managing plant pests and diseases.

% Ok, introducing network models. Good, but you need to give a bit more background first on some of the other methods too, this is just too blunt. You have a space before a "."
Network models provide powerful tools in many branches of science, which are applied for generate to . A network is an abstract model composed of a set of nodes or vertices, a set of edges, links or ties that connect the nodes, together with information concerning the nature of the nodes and edges. The nodes usually represent entities and the edges represent their relations. This simple model can be used to describe many kinds of phenomena, such as social relations, technological and biological structures, and information networks.
%-----------------------------
\section*{OBJECTIVES}
%----------------------------

% 1. Rethink the use of the idea of yield "constraints". Also, for the last time, constrain is a VERB. Constraint is a noun. Learn it. Apply it. Use it. Never make the mistake ever again even though you won't be using the term in this context because I told you to rethink it. 
% 1a. The opening sentence makes no sense even if yield constraints were the proper terminology to use.
% 2. Are you sure about addressing how pests and cropping practices relate to yields?
% 3. The third sentence is poorly written and not easily understood.
% 4. The last sentence is incomprehensible. 
% 4a. "The examples"? You are indicating that the examples you have are THE only examples possible. Think back to the in-depth conversation with you about the use of the word "the" when describing things like this and how you should use it.

% Lastly, I don't agree with your "objectives" as you've written them here.
My objectives are to synthesize approaches to characterize the yield constrains under agroecosystem. I also address how pests and cropping practices relate to yields individually and simultaneously. For better understanding, climatic variable can be possibly added. Finally, I propose the examples of new applications and a conceptual framework for exploring the relationships of injuries caused by pests, human activities under different geographic location.

% I see three questions? Which one are you answering of the following?
The networks constructed will attempt to answer the following question. 
\begin{enumerate}

\item How can the rice yield losses be examine from the perspective of networks analysis. What the key factors affect to rice yield productivities? and how variation is in different locations?  
\item What relationships between components of networks as defined in the variables in survey profile data? 
% why is the there a space before the "?"?
\item How are theses relationships affects by different locations ?

\end{enumerate}
% 1. The anticipation of the proposed work won't provide anything other than that, anticipation. Rewrite.
% 2. Second sentence is incomprehensible. Rewrite.
% 3. Third sentence is  incomprehensible. Rewrite.
The anticipation of the proposed work will provide the insights of rice injuries from network inference of rice crop health survey data. Complete an analysis of the crop health survey data using the new network model and provide visualizations and interpretations of the results and make rice crop health recommendations based on findings. It will be very helpful for plant health authorities worldwide, in order to design specific strategies for rice pest and disease management and to limit the impacts of these yield reducing factors.
 

