% introduction of literature review
%===================
\section*{Introduction}
\addcontentsline{toc}{chapter}{Introduction}
%==================
\label{ch:intro}

The applications of network analysis have increased exponentially over the past two decades in various disciplines. Even though documented applications of network analysis in plant pathology are still relatively sparse, network applications in the social science, systems biology and ecology have been increasingly found. \shortciteA{Shaw:2014cka, MoslonkaLefebvre:2011fo, Jeger:2007tn, windram2014network} presented useful concepts and methods of network analysis in the studies related to plant pathology. I review the empirical works that exist and argue that network analysis is a promising approach for exploring questions in the context of plant pathology.

% Describe what you are going to talk about in each section, in order. You end your para back at the first two sections where you started, confusing.
% 
This chapter contains four sections to thoroughly review of network analysis and its applications. In the first sections, I introduced a brief overview of the concepts and methods of network analysis, and I then discussed the unique values of network analysis that are not found in other approaches. In the third section I focused on network analysis into the current applications of plant pathological research, particularly in plant disease epidemiology and molecular plant pathology, which network analysis has been broadly applied, and increasingly documented. Network analysis provides fruitful tools for visualizing, analyzing and understanding complex relationships in the studies of plant disease. For instance, network models of genes or proteins pertaining plant defense mechanisms and network models revealing spatial distribution of plant disease through trade networks were reviewed by \shortciteA{windram2014network} and \shortciteA{Shaw:2014cka}, respectively.
%The last section presented analytical techniques and strategies to apply network analysis for studying pest management. It introduces three strategies of network applications, which are developed and applied in the studies of systems biology and ecology, and concludes the brief discussions of potential applications to the studies of crop heath management that have yet been undertaken.

\section*{Part I: Network Analysis}
%==========================
\addcontentsline{toc}{chapter}{Network analysis}

\subsubsection{Introduction to network analysis}
Network analysis is used for determining relationships between elements of interest. It offers toolkits for visualizing data in a network model and measuring its properties, and network thinkings \shortcite{PROULX:2005hx}. It has been widely used by various branches of science, such as social science, ecology, biology, computer science, and many others to study the interactions between elements, e.g., the relationships of students in school \shortcite{moody2001race}, species in food webs \shortcite{krause2003compartments}, interactions of genes or proteins in cells \shortcite{guimera2005functional}, or the connections of computer in the network \shortcite{pastor2001epidemic, newman2006modularity}.

\shortciteA{newman2003structure} loosely categorized four types of networks based on different complex data. The first category is social network, representing sets or groups of people forming some patterns of contacts or interactions between them such as the patterns of friendship or business relationships. Analyzing the structure of whole social entities gives us the perspectives from a social network, which enables us to explain the patterns observed. \shortciteA{moody2001race} analyzed the social behaviors in high school students using social network approaches. \shortcite{Kasari2011social} applied network analysis to compare the social relationships and friendships between children with and without autism spectrum disorder (ADS). The second type of network is an information network or knowledge network. The classic example of this network is the network of citations between academic papers \shortcite{newman2003structure}. The articles cited other papers, which have related topics. They formed a citation network that has vertices as articles and direct links as citations. The citation network visualizes the structure and the movement of the information. The third category, technological network, is object connected network, or man--made network which represents a physical connection between objects. This network is mostly applied for illustrating physical structures and systems such as the electrical power grid, the connections of rivers, transport systems, etc. The fourth category of network is a biological network. It represents the biological systems such as genes to genes, genes to protein, protein to protein interactions, which enable biologists understand the connections and interactions between individual constituents including genes, proteins, and metabolites at the level of the cell, tissue and organ to ultimately describe the entire organism system. Biologists use biological networks in various branches of biology at different levels (from a single molecule to an entire organism). For example, \shortcite{yang2014gene, barabasi2004network} studied in the patterns of gene expression in different conditions and different types of cells (normal cells and cancer cells) in order to characterize the genes that change and do not change following the particular conditions; \shortcite{freilich2010large} applied a molecular ecological network analysis to study the communities of soil microorganisms. Networks revealed the complex relationships between microbial species in soils and their communities. Moreover, network analysis enables ecologists to understand ecological properties and predict the ecological roles of species in a soil ecosystem. Although the application of each type of network approach varies, all four categories of networks share a common empirical focus on relational structure and a similar set of mathematical analysis. 

Network analysis can be a powerful tool to study plant disease. \shortciteA{Lefebvre:2011fo, Jeger:2007tn, windram2014network} reviewed the applications of network analysis in botanical epidemiology and molecular plant pathology. Network models were applied to reveal the dynamics of the disease spread, or the plant defense mechanism during plant-pathogen interactions.

\subsection*{Concepts, principles, and methods of network analysis}

% this sentence is unclear: "Network analysis aims the association among nodes rather than the attributes of particular nodes."
A network represents relationships between of elements of interest, which is defined by links (edges) among nodes (vertices). Nodes can be units of interests or studies, and links represent interactions between nodes. Network analysis aims the association among nodes rather than the attributes of particular nodes. In network analysis, networks are defined as any set or set of links between any set or sets of nodes.

% You use "e.g." elsewhere but "for example" 2X here, be consistent. I'd suggest using "e.g." sith[clear]
Network analysis follows three principles. Nodes and their behaviors are mutually dependent, not autonomous; links between nodes can be channels for transmission of both material (e.g, money, disease) and non-material (e.g., information, knowledge, relationship, interaction) and; persistent pattern of association among nodes create structure that can define, enable, or restrict the behavior of a node.  

% Processes is a very generic term to use here, "Flow models are good for evaluating processes", be more specific
Network models have two different organizational structures depending on goals of the representation and analysis \shortcite{borgatti2013analyzing}. Flow models, so-called directed graphs, view the network as a system of pathways along which somethings move such as transportation networks (e.g., of highways, railways and airline), communication networks. Since flows have directions, from an origin to a destination. The processes of movements  are interesting to being modeled as a network model. Analysis of flow models can identify which nodes in the network are more active, or which ones are more important connectors. \shortciteA{Jeger:2007tn, Shaw:2014cka} applied such network models to study plant disease spreads. Architectural models, so-called indirected graphs, are mainly used to determine the structure of the network, seeking to discern whether specific structures lead to similar outcomes, or whether nodes in similar network positions behave in similar ways. Ecological applications related to the ecology and spatial structure of ''community'' tend to be organized and analyzed as architectural models. For example, \shortciteA{Faust:2012dk} studied the networks of soil microbial interactions. Network models can describe how microbial populations change over time, which will require the use of dynamic models of microbial communities. Beyond these basic principles, network analysis enables the calculation of structural properties of nodes, groups, or the entire network.

\textit{Measuring network properties}

A network is made up of nodes and links from relational data. It is constructed from an adjacency matrix, which is obtained from analysis using metric algebra techniques. The row and column headings for an adjacency matrix are identical, listing the names of the components involved in the network. In the simplest case, the cells of the matrix are coded with ''1'' if an link exists between the node or ''0'' if no edge exists. However, a link can be valued. Value indicates a characteristic of the relationship that the research has quantified. The values may be binary, such as whether two friends recognize each other, or variable strength, e.g., the number of mutual friends between two friends. A network link need not to imply positive or cooperative interaction; they can also be a negative or competitive interaction between two individuals.    

The distribution of links in a network suggests two important structural characteristics: centrality (importance) of nodes in the network and division of the network into subgroups. Variants of centrality in a network include degree, closeness, and betweenness. Degree centrality of a node is the sum of the value of the links between that node and every other node in the network. This measure tells us how well-connected a particular node is to the other nodes. Closeness centrality is calculated using the length of the path between a node and every other node. This measure could estimate the time required for information or resources to propagate to a given node in a network. Betweenness centrality corresponds to the number of paths in the network that pass through a particular node, and therefore measures the dependence of a network on a particular node for maintaining connectedness \shortcite{Toubiana:2013cv}. \shortciteA{Deng:2012do, newman2003structure, Toubiana:2013cv} are recommended references for descriptions of the theory and uses, as well as the formal calculation of these measures.


%==========================
\section*{Part II: The unique values of network analysis}
\addcontentsline{toc}{chapter}{The unique values of network analysis}
%==========================

There are four key points that will help to understand network analysis 1) how it differs from traditional approaches to social science research; 2) how it relates to those traditional approaches; 3) how networks are constructed, manipulated and measured; and 4) what value network analysis offers beyond traditional approaches.

The first point of network analysis is that there are two types of data resenting in the network graphs; technical and rational data. The first is data about the actors or variables being studied referring to attributes. Attributes describe characteristics of individual actors or variables, for example their race, income or physical location, and are the primary variables considered in traditional approaches. The second type of data is relational data - that is, data about the relationships between individual nodes. For example, \shortciteA{Lazega} represented the network model of cooperation among lawyers in three law firms, through the exchange of various type of resources among them. This model consisted of over 70 lawyers in three different offices in three different cities. Rational data reflected to resources exchange, and additional attribute information were included type of practice, genders, and seniority of each lawyers.
was recorded for each lawyers , including type of practice, genders, and seniority.

Relationships are also referred to as edges (links) in network analysis. Edges cannot be attributed to any single actor. Rather, edges only exist between nodes. This leads to the second point about network analysis that it requires a different conceptual approach. Because edges only exist between nodes, it is useful to think of edges existing in a separate dimension from nodes, who are anchored in physical space. This dimension is sometimes referred to as relational space. To visualize the difference, think of someone far away with whom you correspond regularly, say using a phone, email, or Facebook. Even though the two of you are not physically close, you have a strong relationship. The two of you are distant in physical space but close in relational space. This notion of relational space is in part what means when he refers to the space of flows as something distinct from the space of places \shortciteA{Castells}.

The third point that distinguishes network analysis from other approaches is that it involves different methods of analysis. Because traditional research methods consider variable attributes in a wide variety of statistical analyses such as measures of center (e.g., mean, median, etc) and dispersion (e.g., standard deviation, range etc.), these methods are sometimes referred to as variable analysis, Whereas, network analysis models relational data and to measure various characteristics of network structure. For example, for lawyers data \shortciteA{Lazega}, it is natural to ask to what extend two lawyers that both work with third lawyer are likely to with  each other as well. This notion corresponds to the social network concept of transitivity and can be captured numerically through an enumeration of proportion of vertex triples that form triangles, so-called cluster coefficient. 


The idea that network structure may be correlated with variable attributes and behaviors is the fourth point to consider in comparing network analysis to other approaches. In network analysis, the arrangement of the network in relational space is basically correlated with the behavior and attributes of those variables. For example, in the network created by \shortciteA{Lazega} lawyers of the same firm may share similar attributes such as office location or department, and lawyers in similar roles within that network may share similar behaviors. Basically, conventional approaches measure various attributes of variable (nodes in a network) and attempts to discern something about the relationships between actors (edges in a network) based on those attributes. When the network structure is simple and the differences in node attributes are clear, the conventional analytic approach is sufficient. However when relationships are complex or node attributes are more nuanced, clear answers using conventional analysis may prove elusive. As a result, network analysis offers a tool to help researchers visualize the large network and disentangle some of the relational complexities with in the network, just as cluster analysis and multivariate analysis for help research disentangle the complex data.


\section*{Part II: Networks and Plant Pathology}
\addcontentsline{toc}{chapter}{Networks and Plant Pathology}

% Plant pathology embrace? Unclear. Do you mean Plant pathologists? If so, you need "have" in there too.
% Revise this whole sentence, it's incomprehensible as written.
Recently, a broad expansion of applications of network analysis has occurred across many disciplines over the past decade, and several researchers have evaluated network analysis as a promising tool to study a complex system. Plant pathologists also have used network analysis for their research. \shortciteA{Lefebvre:2011fo, Jeger:2007tn, windram2014network} supported that network analysis can be fruitful models in many applications relevant to plant pathology because of With its generality and flexibility. For example, the network of main fresh cut flowers movements among European countries was determined the likelihood of introduction of new pathogens and other organisms associated with plants \shortcite{Lefebvre:2011fo}, and plant-pathogen interaction network models were applied to present plant defense mechanisms \shortcite{windram2014network}.

% you're reviewing studies that haven't completed? I don't understand what you mean by "have begun to integrate them" seems like either they have integrated them or they have not.
The development of network analysis challenges conventional approaches to uncover rational complexities of plant pathology studies. Two fields of research relevant to plant pathology present particularly strong growth and prove that network analysis has significant potential to augment traditional analysis methods. The first is plant disease epidemiology, which investigates questions related to plant disease spread. The second is plant molecular biology, which investigates questions related to biological networks.


\subsection*{Using Network analysis to understand plant disease spread}

Network analysis challenges conventional approaches of studies in plant disease epidemiology. Network approaches for spread of pathogens through trade network typify analytical risk assessment.
For example, a structure analysis of network modeled the process of plant disease spreads.

\textit{\textbf{Network models of epidemic development}}

%As I mentioned above, network models applied for this study are categorized as flow models or directed networks. 
The idea of plant epidemics is that the probability of infection embedded in the connection or the contact patterns between susceptible/infected plants, and it forms as the networks.
\shortcite{Lefebvre:2011fo} showed a network model of epidemic development (susceptible-infected-susceptible model) in a directed network. In the network model, vertices were represented plant, and theirs attributes were presented the infectious status (healthy or infected). The epidemic is started at a single node, then nodes with a connection from the starting infected node will be infected at the next time step with a certain probability of transmission. In turn, already infected nodes will be infected at the next time step depending on their infection status and on a certain probability of persistence. The probability of infection transmission is the same for all connections between infected nodes and susceptible nodes over times. Similarly, the probability of infection persistence is the same for infected nodes in a certain network replicate. For each network structure, the two probabilities of persistence and transmission define an epidemic threshold, which is independent of the starting node of the epidemic. This epidemiological model does not result in either susceptible or infected nodes, as nodes will have a infection status along a continuum. Key quantities for epidemiological dynamics in network were reviewed in \shortciteA{Lefebvre:2011fo}.

\textit{\textbf{Analysis of plant trade network}}

\textit{Phytophthora ramorum} epidemic networks in the horticultural trade is an example of application of network models in the study of plant disease spreads. Simulations of spread of \textit{P. ramorum} in different network structure was found that epidemic threshold, the boundary between a no epidemic an epidemic outcome, is significantly lower for scale-free network, a network is dominated by a small number of nodes with many connections, compared to local, random and small-world network structure. Modeling suggested that was possible to control an epidemic by changing the structure of network, without having to decrease the probability of infection persistence at a nursery site and/or of transmission between sites.

Regardless of the network structure and connectivity level, epidemic threshold is negatively correlated to the correlation coefficient between link in and out nodes, \shortcite{Moslonka2008}. In presence of high-connected nodes, the most effective way to control disease spread s to move from two-way to a one-way network, i.e. from network where overall there is positive correlation among links-in and -out to one where the correlation or negative. In practice this would mean that a nursery network would be dominated by major node, which receive plant materials from many production sites but supply a relatively few retail sites, or by major nodes which received plant materials from a few production sites but supply many retail sites. The scenario where there are major nodes which both receive plant material from many production sites and supply many retail sites is the most problematic control of this control s target towards such hubs. \textit{P. ramorum}, these epidemic size would be the number of nurseries/retail centers with more than a certain proportion of plants infected, or the overall number of infected plant in all nurseries/retail centers. Simulations showed that the number of equilibrium. This correlation increase with connectivity level for all the structures investigated and under lines the importance of targeted control towards node with more connections than others \shortcite{Moslonka2010}. 

The last point, the modeling of disease spread in small-size directed networks showed that increasing the proportion of wholesalers (i.e., traders without a preponderance of incoming or outgoing links) tends to decrease the epidemic threshold in no-scale free networks (local ,random, and small-world ones) The opposite result is obtained for the proportions of produces and retails. Scale free network appear instead to be immune to changes in these hierarchical categories, as the epidemic threshold in this case is governed by the presence of hub rather than by the features of the majority of nodes in the network.

\textit{\textbf{Network models to design strategies of plant disease management}}

% Revise first sentence. Unclear.
% Second sentence is not a sentence.
% This is an important para, but it's unclear what you're trying to communicate. Clarify, please.
Due to globalization, increases in trade among countries affect to plant health. Network models presenting flows plant disease epidemics through plant trade network and develop strategies of plant disease management. Hubs or highly connected nodes represent locations or countries, where import and export plants or plant parts can be found in theses network models. Because they take into account the likelihood of pathogen actually infecting along particular links following the network concepts, nodes are obviously targets for reduce disease flows in the scenarios that network model presented. Strategies for disease management may aim to remove them from the network or limit their geographic connectedness to try to alter the structure so as to slow and limit pathogen spread\shortcite{Shaw:2014cka}. Alternatively, strategies may focus on prevention of disease spreads by throughout monitoring on hubs and highly connected nodes. For trade networks, \shortciteA{Lefebvre:2011fo} suggested that it would be sensible to place quarantine efforts on hubs or on connections between major hubs.

However, wherever effort is placed, an increase in quarantine effort is needed to keep the rate of flow of pathogens across trade links constant as the trade through links increases \shortcite{Shaw:2014cka}. For instance, trade network of plants and plant products across the world and within countries give the picture on how to be able to control the flow of pathogens. The strategy should be designed by focusing on links to and from hubs, nodes which have high degree of connectivity would increase efficiency to achieve control plant pathogen spreads. To cope with increasing volumes in trade of potential infected plants, this insight may be very helpful for plant heath authorities target at the traders who have high connection activities or find the major pathways. The control of disease or quarantine can be made more efficient and effective \shortcite{dehnen2010structural, Lefebvre:2011fo, Shaw:2014cka}.

Additionally, \shortciteA{Lefebvre:2011fo} showed the good examples, which are co-occurrence network of the \textit{Phytophthora ramorum} infected plant genera different environment. The networks may be helpful in identifying host taxa playing a important role in spreading a certain disease in the seminatural environment, in crop plants, and plants in the trade. Combining genetic network analysis and data on trace forward and trace back on movement of plants nursery trade supported to identified confidentially \textit{P. ramorum} migration. From this approach, it was clear that the pathogen was introduced originally from nurseries, which \textit{P. ramorum} populations in nurseries are genetically ancestral to all Californian forest populations.

\subsection*{Using Network analysis to understand molecular plant pathology}

Within the fields of molecular plant pathology, potential applications of network analysis include analysis of a protein’s or gene’s functions and their interactions in order to understand mechanisms of plant-pathogen interactions.

\textit{\textbf{Presenting biological data with network model}}

\shortciteA{wu2007gene, Lefebvre:2011fo} gave the example of the network model built from the gene-for-gene relationships between rice and various avirulence genes of the pathogen \textit{Xanthomonas oryzae pv. oryzae} causing bacterial leaf blight of rice. Nodes were represented isogenic lines of rice, and weighted edges reflected the number of shared genes with high resistance (with respect to avirulence genes) in the two isogenic lines of rice. For plant breeder, this graph can help in identifying particularly promising genes for developing plant resistant to pathogens. 

\textit{\textbf{Network analysis to study biological systems}}
% First sentence is unclear
% Second sentence is unclear

Network analysis offers tools to visual the myriad information and analyze the complex relationships. \shortciteA{Zheng2013} used coexpression network inference to investigate plant immunity. The network modeled from transcriptome data sets of citrus infected with \textit{Candidatus} Liberibacter asiaticus bacterium. This network revealed contained hub genes (genes may have similar functions), potentially key components of defense mechanisms, and novel genes that liked to defense. 

The main use of coexpression networks using large collections of static expression data is gene discovery. However, many biologists have attempted to construct differential networks from large scale biological data sets with different conditions or targeted experiments (control and treatment group) \shortcite{windram2014network}. \shortciteA{Lu:2013hga} constructed the networks of soil fungal community. The fungal networks with different condition, yield-invigorating and yield-debilitating soils under prolonged potato monoculture were compared. The result showed that \textit{Sordariales} and \textit{Hypocreales} were major affected phylogenetic groups. Network analysis enabled to identify the key elements and their relationships that need to be bridged to overcome problems of relational complexity.

% First sentence is unclear. Revise
% Second sentence is incomprehensible. Revise.
% Does UPLB require the Oxford comma? If so use it, if not and you're not going to use it, don't use it. You waffle, sometimes you use it but mostly you don't.
The network approach focuses on components of networks and their relationships that cannot be created from observing individual nodes alone. Recently, biologists have attempted to understand the nature and consequence if biological complexity using network analysis to understand the network structure of biological. For instance, networks can be constructed from available data for a certain plant pathogen from multiple locations/hosts based on the similarity among the pathogen strains.

% Entire para is incomprehensible and mish-mashed. Revise for clarity and grammar.
Network analysis can contribute to our understanding of biological mechanism and interaction. The network composition and interdependent relationship in biological processes affected to the topologies of network. Network analysis is not needed for simple assessment of network composition; that is, to measure the relative levels of key components in the processes. However, what it does offer is the ability to identify and compare the structural positions of individuals and their relationships. Systematically and simultaneously analyzing network composition and structure provides much deeper insights in holistic view. In the review of \shortciteA{mukhtar2011independently} plant-pathogen interaction network revealed the network of interactions of novel \textit{Arabidopsis} protein-pathogen effectors, provided evidence that pathogen effectors target a limited number of host immune proteins, and demonstrated that effectors from very distantly related pathogens interact with the same host proteins. 

\section*{Summary}
\addcontentsline{toc}{chapter}{Summary}
% this literature? I would say it's a literature review that you've written.
% " Building on this foundation the literature ten identified two branches of plant pathology which network analysis has been identified as particularly useful, and examine theses." unclear. ten what?
% plant disease epidemiology or botanic epidemiology? stick with one term.
% You are saying the literature did something then you say you did something. Be clear. Be consistent.
This literature review presented a brief introduction of network analysis and concise concepts and methods. It also attempted to point out unique aspects and values that could not find in traditional approaches. To convince that network analysis can be applied in plant pathology, literature review showed the applications of network analysis to studies related to plant disease. 

This literature review presented a brief introduction to the concepts and methods of network analysis. Networks have four types, social network, information, technology network, and biological network. Even though four types of networks are described and applied in different context, they share a common empirical focus on relational structure and a similar set of mathematical analyses. Node or vertex, link or edge. Network models are cable of presenting unique values, which are traditional approaches can not present. It attempted to position network analysis as both a unique perspective and unique methodology.

% First sentence is unclear. Revise.
% Understanding not understating
% "the literature suggests that further use of networks analysis concepts and methods enable us the holistic understand the flow of disease spread" unclear, revise
% "The literature of molecular plant pathology offers two challenge". Two is plural yet challenge is singular. Correct.
% Check your entire para here. The words used are incorrect. Plurality is wrong.
Network concepts and methods were applied to two broad studies particularly of plant disease. Network analysis was utilized to model plant disease spreads and to analyze the structure of models. The results enabled us to understand the directions and processes of disease spread. Additional, they   predict how, and improve the implementation of plant disease policy. In molecular plant pathology the literature review showed two challenges of network applications. The fist challenges is to apply network to model large and complex biological dataset. Another challenge is consideration network structure to understand biological system. Emergent properties of network structure influences may be identified, measured and analyzed to yield better explanations of the experiments being observed.
% Punctuation in this para and elsewhere in this whole section needs work. You do not capitalize after a comma. Only on full stops.
% revise para. unclear.
% The last sentence is a good way to close this section.

While number of documented plant pathological studies using network analysis are sparse, the literature presented in this review showed a clear and compelling case for plant pathologists to expand the understanding, use network concept and methods. Network analysis concepts and methods augment existing approaches and provide tools for exploring complex relationships, which has been widely acknowledged as influential but difficult to measure using traditional methods. With concepts and approaches giving generality and flexibility, networks can potentially model survey data. 

% eos
