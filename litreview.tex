\documentclass[12pt, oneside]{report} 
\usepackage[a4paper]{geometry}
%\usepackage{fontspec}
%\setmainfont{Times New Roman}
%\setmainfont{Courier}
\usepackage{pdflscape}
\usepackage[round]{natbib}  % the bibliography
\usepackage{hyphenat} % for hp hyenation
\usepackage{enumitem} % enumerate package
\usepackage{setspace} % format the line spacing
\linespread{2}
\usepackage{siunitx}
\usepackage{amsmath,amssymb,amstext} % Lots of math symbols and environments
\usepackage{graphicx} % For including graphics N.B. pdftex graphics driver 
%\usepackage{hyperref}
%=== Note taking ====
\usepackage[colorinlistoftodos,prependcaption,textsize=tiny]{todonotes}


%======================================================================
%   L O G I C A L    D O C U M E N T -- the content of your thesis
%======================================================================
\begin{document}
% 1. You have spelling errors everywhere that my spellcheck is catching. There is absolutely no reason for you to submit anything to me with these types of errors. I can understand "scares" and "scarce" but when it's just plain misspelled and the spellchecker catches it, you should be correcting it, not me!

% 2. Go through and remove every reference to the RKB. This is a LITERATURE review. I expect you to read papers. Not the RKB. I want real scientific papers, not the RKB. Yes, it's reliable, no it's not a peer-reviewed article, which is what this is supposed to be reviewing.
% 3. This is a review of the literature, which means you are reading and citing what you've read. You have far too much that is not cited. Too many paragraphs with no citations. What did you read to get this information?
%=====================================================================================

The world's population is growing rapidly.  It reached 6 billion people in 1999 and is anticipated to reach 8.1 billion in 2025 and 9.6 billion in 2050 \citep{Alexandratos_2012_World}.  Our long-term ability to meet growing needs for food seems uncertain.  Thus, one of the greatest challenges is increasing food production in a sustainable way so that everyone can have adequate food and proper nutrition without over-exploiting the Earth's ecosystems. 

Rice is predominantly produced in Asia, so much so that thirty--one percent of the rice harvested globally comes from Southeast Asia alone \citep{OECD_2012_Agricultural}. The highest levels of productivity are found in irrigated areas, the most intensified rice production systems. Farmers can grow more than one rice crop per year here. Approximately 45 percent of the rice growing country in Southeast Asia is irrigated, with the largest irrigated areas been found in Indonesia, Vietnam, Philippines and Thailand \citep{Mutert_2002_Developments}. In South Asia, the two major rice-growing countries are India and Bangladesh. India has the largest rice growing area globally, about 43 million hectares, and contributes 25 percent of global rice production. Combined, rice production in South and Southeast Asia contributes around half of global rice production. If rice production in South and Southeast Asia is threatened, it will significantly affect global rice production.

Pests in rice production are significant yield reducing factors globally. \cite{Oerke_2005_Crop} estimated that weeds, animal pests, and disease caused losses around 10.2, 15.1 and 12.2 percent of global rice production, respectively. In most Asian countries, rice yields average 3-5 t/ha.  One recent survey estimated that between 120 and 200 million tons of grain are lost yearly to insects, diseases, and weeds in rice fields in tropical Asia \citep{Willocquet_2004_Research}. The mean region-wide rice yield loss due to pests was estimated at 37 percent \citep{Savary_2000_Quantification}.

In crop fields, pests or so-called biotic constraints can be defined as organisms that cause plant injuries and lead potentially to economic losses. Among the pests that attack rice are microorganisms (viruses, mycoplasmas, phytoplasmas, bacteria, oomycetes, and fungi) that can cause diseases, parasitic plants, weeds, invertebrates (insects, mollusks), and even vertebrates such as rats and birds can cause serious damages.


\bibliographystyle{apalike2}
\cleardoublepage
\begin{singlespace}
\renewcommand{\bibname}{LITERATURE CITED}
\bibliography{reference}
\end{singlespace}
\end{document} 
% eos



