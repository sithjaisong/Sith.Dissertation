%=========================
%INTRODUCTION
%=========================
%\section*{Introduction}
\addcontentsline{toc}{chapter}{Introduction}
% 1. Why are you talking in past tense in your opening sentence? I already commented about being consistent in your verb tenses.
% 2. Sentence 2 should be part of sentence 1.
% 3. "at irrigated areas" is awkward. It should be "in irrigated areas"
% 4. No "," after "approximately".
% 5. Your punctuation in the entire para is a shambles.
% 6. This para is out of place. It needs to be in your M&M for the section.

Rice was dominantly produced in Asia. 31 percent of the global rice harvested comes from Southeast Asia \shortcite{oecd2012oecd}. The highest levels of productivity are found at irrigated areas, where is the most intensified production system. Farmers can grow rice more than one crop per year here. Approximately, 45 percent of the rice growing area in Southeast Asia is irrigated, with the largest areas being found in Indonesia, Viet Nam. Philippines and Thailand \shortcite{mutert2002developments}. In South Asia, major rice-growing countries are India and Bangladesh. India has the largest rice growing area (approximately 43 million hectares) in the world and contributes 25 percent of global rice production. 

The pests and diseases to global rice production are significant yield reducing factors. \shortciteA{OERKE:2006ct} estimated that rice pests potential caused losses around 37 percent of global rice production. Additionally, future rice production will need to grow by 2.4 percent per year in order to meet the demands of a growing population \shortcite{Ray:2013by}. Addressing these yield-reducing factors is essential food security not only in rice consuming societies, but also global societies. 

% 1. first sentence is incomprehensible
% 2. Savary sentence grammar is incorrect. Plurality, again.
% 3. "This study" sentence, punctuation
% 4. "Moreover, most previous" indicates more than one item. Plurality, again.
% 5. Overuse of ","

Designing the programs of pest and disease control for all kind of pests at anywhere seems complicated because pests are diverse and have an effect on human activities. To achieve this, relationships between pests and human activities must be studied. \shortciteA{Savary:2000ws} characterized the patterns of pest injuries and the patterns of cropping practices, and how they related to each others. This study suggested that pest management strategies should be developed according to the patterns of cropping practices, and production situations. However, interactions among pests, cropping practices, and environments are difficult to elucidate. Moreover, most previous analytical technique can not be used to reveal their changes across locations or time, and interactions among cluster members, which could be important to shape pest management. 

% 1. First sentence grammatically incorrect/awkward
% 2. Second sentence, awkward, does not flow
% 3. I don't understand "on a school"? Were the individuals standing on the roof?
% 4. You need an "or" in your last sentence so it flows better.
Network analysis is prove to be a powerful tool in revealing the interactions among entities within a complex system and applied for many branches of science. The heart of network analysis is a network model, which is an abstract model composed of a set of nodes or vertices and a set of edges, links or ties connect the nodes. Nodes usually represent entities and the edges represent their relations. This simple model can be used to describe many kinds of phenomena, such as social relations on a school \shortcite{moody2001race}, ecological relationships of species in food web \shortcite{krause2003compartments}.

%-----------------------------
\section*{OBJECTIVES}
%----------------------------

% 1. Your first question is grammatically incorrect.
% 2. Your second question is grammatically incorrect.
% 3. Your second question is grammatically incorrect.
% 4. Your last sentence is not a sentence or if it is, I'm extremely unclear on where it starts based on your punctuation.

The objectives of this research is to answer these questions; (a) How relationships among components of injuries and cropping practices are?, (b) How are the changes of theirs relationship under different production environments, or across seasons?, and (c) How their relationships affect to yield gains? by network analysis approaches, using cropping practice and injury profile network as model based on crop health survey data.

% 1. Sentence is too long.
% 2. I have no idea what you're trying to say here?

The crop health survey data were modeled to cropping practice and injury profile network in anticipation of being helpful for plant health authorities, in order to design specific strategies for rice pest and disease management and to limit the impacts of these yield reducing factors.  


% eos
