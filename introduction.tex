%=========================
%INTRODUCTION
%=========================
%\section*{Introduction}
\addcontentsline{toc}{chapter}{Introduction}
% Comment on whole document. Once again. I will not say it again. If UPLB wants you to use the Oxford comma or not use, then follow their guidelines. If they do not give guidelines then chose one style and stick with it. You change between sections, paragraphs and sentences.

Pests and diseases to global rice production are significant yield reducing factors. \shortciteA{OERKE:2006ct} estimated that rice pests potentially caused losses around 37 percent of global rice production. Additionally, future rice production will need to grow by 2.4 percent per year in order to meet the demands of a growing population \shortcite{Ray:2013by}. Addressing these yield-reducing factors is essential for food security not only in rice consuming societies, but also for other societies globally.

% 1. which yield reducing factors caused 30% of yield losses or do you mean, "showed that yield reducing factors caused 30 percent yield losses."? They are very different things.

% I integrated paragraphs 2 and 3 as I previously instructed...
Rice is predominantly grown in Asia. So much so that thirty--one percent of the rice harvested globally comes from Southeast Asia \shortcite{oecd2012oecd} alone. The highest levels of productivity are found in irrigated areas, the most intensified rice production system. Farmers can grow more than one rice crop per year here. Approximately 45 percent of the rice growing area in Southeast Asia is irrigated, with the largest irrigated areas being found in Indonesia, Vietnam, Philippines and Thailand \shortcite{mutert2002developments}. In South Asia, the two major rice-growing countries are India and Bangladesh. India has the largest rice growing area globally, approximately 43 million hectares, and contributes 25 percent of global rice production alone. Combined, rice production in South and Southeast Asia contributes around half of global rice production. If rice production in South and Southeast Asia is threatened, it will significantly affect global rice production. 

Yield losses in Asian rice have previously been studied by the International Rice Research Institute (IRRI). A 10-year study conducted by IRRI and its partners \shortcite{Savary:2000vr} in over 300 farmers'�� fields across Southeast and South Asia, showed that yield reducing factors caused 30 percent of yield losses. Rice is not threatened by many pests and diseases in a single season. This combination of injuries caused by pests and diseases can be thought of as a crop health syndrome. The combinations of injuries depend on the production situation (\textit{i.e.}, the cultural practices and inputs used to produce a rice crop) as a range of agroecosystem \shortcite{savary2006quantification}.

%% 3. Last sentence is a bit unclear. Revise.
Nowadays, developing the strategies of pest and disease management takes into account sustainability, production efficiency, and environmental protection \shortcite{Mew:2004kh}. To achieve this, interactions between pests and human activities must be studied. A survey may provide the necessary data and adequate methods for analyzing survey data can produce preliminary information on their behaviors including major interactions \shortcite{savary1995use}. \shortciteA{Savary:2000vr} concluded that the observed injury profiles (\textit{i.e.}, the combination of disease and pest injury that may occur in a given farmer's field) were strongly dependent on production situation.  The authors discussed that pest management strategies should be developed according to the patterns of cropping practices, and production situations. However, interactions among pests, cropping practices, and environments under different locations or over time are difficult to elucidate, which they are important to design the strategies of pest management.

To help visualize and understand these interations, network analysis seems to provide a promising tool for revealing the interactions among entities within a complex system. It has been applied for many branches of science (\textit{e.g.}, social science, computer science, and biology). A network model is an abstract model composed of a set of nodes or vertices and a set of edges, links or ties connected to the nodes. Nodes usually represent entities and the edges represent their relations. For example, an ecological network of a food web presents nodes as species \shortcite{krause2003compartments} and edges as ecological relationships, or consider a social network of students in the school present where nodes are students and edges are friendships \shortcite{moody2001race}.

%-----------------------------
\section*{OBJECTIVES}
\addcontentsline{toc}{chapter}{Objectives}
%----------------------------

My overall objective is to develop network approaches and apply them to analyze crop health survey data. My first objective for this research is to develop the network model based on crop health survey data and characterize relationships among components of injuries and cropping practices. My second objective is to compare the differential relationships of network models under different seasons or locations. The third objective is to apply network analysis to compare differential interactions in networks under successive (from low to high) levels of estimated actual yields.

Once network models based on crop health survey data are constructed, they will be helpful for plant health authorities and the people who related to crop protection especially for rice. The models will support them to design specific strategies for rice pest and disease management and to limit the impacts of these yield reducing factors.  

% eos