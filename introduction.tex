%=========================
%INTRODUCTION
%=========================
%\section*{Introduction}
\addcontentsline{toc}{chapter}{Introduction}
% 1. Threats do not reduce yields. Pests and diseases do. Rewrite.
% 2. Cite Oerke for actual and potential losses to bolster your case.
% 3. Your second sentence is out of place with the opening. You need to work on making the paragraph flow.
% 4. I'd argue that it's important for not just rice consuming societies but globally due to the interconnectedness of our modern societies.

Rice was dominantly produced in Asia. 31 percent of the global rice harvested comes from Southeast Asia \shortcite{oecd2012oecd}. The highest levels of productivity are found at irrigated areas, where is the most intensified production system. Farmers can grow rice more than one crop per year here. Approximately, 45 percent of the rice growing area in Southeast Asia is irrigated, with the largest areas being found in Indonesia, Viet Nam. Philippines and Thailand \shortcite{mutert2002developments}. In South Asia, major rice-growing countries are India and Bangladesh. India has the largest rice growing area (approximately 43 million hectares) in the world and contributes 25 percent of global rice production. 

The pests and diseases to global rice production are significant yield reducing factors. \shortciteA{OERKE:2006ct} estimated that rice pests potential caused losses around 37 percent of global rice production. Additionally, future rice production will need to grow by 2.4 percent per year in order to meet the demands of a growing population \shortcite{Ray:2013by}. Addressing these yield-reducing factors is essential food security not only in rice consuming societies, but also global societies. 

% first sentence is incomprehensible
% the second sentence is an incomplete thought, untie what?
% grammar in the third sentence
% fourth sentence, if they have not been insufficient then why would you do this? Obviously they succeeded
% season is a noun, not an adjective

Designing the programs of pest and disease control for all kind of pests at anywhere seems complicated because pests are diverse and have an effect on human activities. To achieve this, relationships between pests and human activities must be studied. \shortciteA{Savary:2000ws} characterized the patterns of pest injuries and the patterns of cropping practices, and how they related to each others. This study suggested that pest management strategies should be developed according to the patterns of cropping practices, and production situations. However, interactions among pests, cropping practices, and environments are difficult to elucidate. Moreover, most previous analytical technique can not be used to reveal their changes across locations or time, and interactions among cluster members, which could be important to shape pest management. 

% Ok, introducing network models. Good, but you need to give a bit more background first on some of the other methods too, this is just too blunt. You have a space before a "." sith[clear]
Network analysis is prove to be a powerful tool in revealing the interactions among entities within a complex system and applied for many branches of science. The heart of network analysis is a network model, which is an abstract model composed of a set of nodes or vertices and a set of edges, links or ties connect the nodes. Nodes usually represent entities and the edges represent their relations. This simple model can be used to describe many kinds of phenomena, such as social relations on a school \shortcite{moody2001race}, ecological relationships of species in food web \shortcite{krause2003compartments}.

%-----------------------------
\section*{OBJECTIVES}
%----------------------------

% 1. Rethink the use of the idea of yield "constraints". Also, for the last time, constrain is a VERB. Constraint is a noun. Learn it. Apply it. Use it. Never make the mistake ever again even though you won't be using the term in this context because I told you to rethink it. 
% 1a. The opening sentence makes no sense even if yield constraints were the proper terminology to use.
% 2. Are you sure about addressing how pests and cropping practices relate to yields?
% 3. The third sentence is poorly written and not easily understood.
% 4. The last sentence is incomprehensible. 
% 4a. "The examples"? You are indicating that the examples you have are THE only examples possible. Think back to the in-depth conversation with you about the use of the word "the" when describing things like this and how you should use it.

The objectives of this research is to answer these questions; (a) How relationships among components of injuries and cropping practices are?, (b) How are the changes of theirs relationship under different production environments, or across seasons?, and (c) How their relationships affect to yield gains? by network analysis approaches, using cropping practice and injury profile network as model based on crop health survey data.

% 1. The anticipation of the proposed work won't provide anything other than that, anticipation. Rewrite.
% 2. Second sentence is incomprehensible. Rewrite.
% 3. Third sentence is  incomprehensible. Rewrite.

The crop health survey data were modeled to cropping practice and injury profile network in anticipation of being helpful for plant health authorities, in order to design specific strategies for rice pest and disease management and to limit the impacts of these yield reducing factors.  



 

