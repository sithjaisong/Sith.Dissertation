%=========================
%INTRODUCTION
%=========================
%\section*{Introduction}
\addcontentsline{toc}{chapter}{Introduction}
% 1. Why are you talking in past tense in your opening sentence? I already commented about being consistent in your verb tenses.
% 2. Sentence 2 should be part of sentence 1.
% 3. "at irrigated areas" is awkward. It should be "in irrigated areas"
% 4. No "," after "approximately".
% 5. Your punctuation in the entire para is a shambles.
% 6. This para is out of place. It needs to be in your M&M for the section.

Pests and diseases to global rice production are significant yield reducing factors. \shortciteA{OERKE:2006ct} estimated that rice pests potentially caused losses around 37 percent of global rice production. Additionally, future rice production will need to grow by 2.4 percent per year in order to meet the demands of a growing population \shortcite{Ray:2013by}. Addressing these yield-reducing factors is essential for food security not only in rice consuming societies, but also for other societies societies globally.

% 1. The sentence starting with "A survey" doesn't need a ","
% 2. Did Serge just imply or did he state that management strategies should be developed according to patterns?
% 3. Last sentence is a bit unclear. Revise.

Nowadays, developing the strategies of pest and disease management takes into account sustainability, production efficiency, and environmental protection \cite{Mew:2004kh}. To achieve this, interactions between pests and human activities must be studied. A survey may provide the necessary data, and adequate methods for analyzing survey data can produce preliminary information on their behaviors including major interactions \shortcite{savary1995use}. \shortciteA{Savary:2000vr} concluded that the observed injury profiles (i.e., the combination of disease and pest injury that may occur in a given farmer’s field) were strongly dependent on production situation. It was implied that pest management strategies should be developed according to the patterns of cropping practices, and production situations. However, interactions among pests, cropping practices, and environments are difficult to elucidate. Moreover, most previous analytical techniques can not be used to reveal their changes across locations or time, which they are important to design the strategies of pest management. 

Network analysis provides a promising tool for revealing the interactions among entities within a complex system. It has been applied for many branches of science (e.g., social science, computer science, biology). A network model is an abstract model composed of a set of nodes or vertices and a set of edges, links or ties connected to the nodes. Nodes usually represent entities and the edges represent their relations. For example, an ecological network of a food web presents nodes as species \shortcite{krause2003compartments} and edges as ecological relationships, or consider a social network of students in the school present where nodes are students and edges are friendships \shortcite{moody2001race}.

%-----------------------------
\section*{OBJECTIVES}
\addcontentsline{toc}{chapter}{Objectives}
%----------------------------

% 1. Confused. I thought that this was basically a proposal for your disseration. Why are your goals written in past-tense? Are they not present or future, you've not completed the work yet, unless I'm unaware.
% 2. You don't need commas before every "and", I've seen this at least twice in this section alone
% 3. Third objective, different levels of yield gains? I'm unclear on what you mean here. Yield gains have a very specific meaning, how are you studying them?

My overall objective was to develop network approaches, and apply them to analyze crop health survey data. My first objective for this research was to develop the network model based on crop health survey data and characterize relationships among components of injuries and cropping practices. My second objective was to compare the differential relationships of network models under different seasons or locations. The third objective was to compare differential patterns of their relationships at different levels of yield gains.

% 1. Again, verb tense. Is this work that you've already done? If so, why are you submitting this to the graduate school?
% 2. Grammar in first sentence and second sentence

Once network models based on crop health survey data were constructed, they can be helpful for plant health authorities, and the people who related to crop protection especially for rice. The models support them to design specific strategies for rice pest and disease management and to limit the impacts of these yield reducing factors.  

% eos
