%=========================
%INTRODUCTION
%=========================
%\section*{Introduction}
\addcontentsline{toc}{chapter}{Introduction}
% 1. Why are you talking in past tense in your opening sentence? I already commented about being consistent in your verb tenses.
% 2. Sentence 2 should be part of sentence 1.
% 3. "at irrigated areas" is awkward. It should be "in irrigated areas"
% 4. No "," after "approximately".
% 5. Your punctuation in the entire para is a shambles.
% 6. This para is out of place. It needs to be in your M&M for the section.

Pests and diseases to global rice production are significant yield reducing factors. \shortciteA{OERKE:2006ct} estimated that rice pests potentially caused losses around 37 percent of global rice production. Additionally, future rice production will need to grow by 2.4 percent per year in order to meet the demands of a growing population \shortcite{Ray:2013by}. Addressing these yield-reducing factors is essential for food security not only in rice consuming societies, but also for other societies globally.

Rice is predominantly grown in Asia. Thirty--one percent of the rice harvested globally comes from Southeast Asia \shortcite{oecd2012oecd}. The highest levels of productivity are found in irrigated areas, the most intensified rice production system. Farmers can grow more than one rice crop per year here. Approximately 45 percent of the rice growing area in Southeast Asia is irrigated, with the largest areas being found in Indonesia, Vietnam, Philippines and Thailand \shortcite{mutert2002developments}. In South Asia, major rice-growing countries are India and Bangladesh. India has the largest rice growing area (approximately 43 million hectares) in the world and contributes 25 percent of global rice production. 

% Add a paragraph tying your first and second paragraphs together. You've said how important pests and diseases are and how important rice is in SE and S Asia. Bring it home in paragraph 3 with why what you're doing is important. It does not need to be long or complex. Just state why this work is necessary and what the big picture impact could be of this work.

Rice production in South and Southeast Asian contributes around half of global rice production. If rice production in South and Southeast Asia is threaten, consequently It will significantly affect to global rice productions. A 10-year study conducted by IRRI and its partners \shortcite{Savary:2000vr} in over 300 farmers’ fields across Southeast and South Asia showed that yield reducing factors caused 30 percent of yield losses. Rice is not threatened by one but by many pests in a season. A combination of injuries caused by pests and diseases can be thought of as a crop health syndrome. The combinations of injuries depend on the production situation (\textit{i.e.}, the cultural practices and inputs used to produce a rice crop) as a range of agroecosystem \shortcite{savary2006quantification}.

%% 3. Last sentence is a bit unclear. Revise.
Nowadays, developing the strategies of pest and disease management takes into account sustainability, production efficiency, and environmental protection \shortcite{Mew:2004kh}. To achieve this, interactions between pests and human activities must be studied. A survey may provide the necessary data and adequate methods for analyzing survey data can produce preliminary information on their behaviors including major interactions \shortcite{savary1995use}. \shortciteA{Savary:2000vr} concluded that the observed injury profiles (\textit{i.e.}, the combination of disease and pest injury that may occur in a given farmer’s field) were strongly dependent on production situation.  The authors discussed that pest management strategies should be developed according to the patterns of cropping practices, and production situations. However, interactions among pests, cropping practices, and environments under different locations or over time are difficult to elucidate, which they are important to design the strategies of pest management.

Network analysis provides a promising tool for revealing the interactions among entities within a complex system. It has been applied for many branches of science (e.g., social science, computer science, biology). A network model is an abstract model composed of a set of nodes or vertices and a set of edges, links or ties connected to the nodes. Nodes usually represent entities and the edges represent their relations. For example, an ecological network of a food web presents nodes as species \shortcite{krause2003compartments} and edges as ecological relationships, or consider a social network of students in the school present where nodes are students and edges are friendships \shortcite{moody2001race}.

%-----------------------------
\section*{OBJECTIVES}
\addcontentsline{toc}{chapter}{Objectives}
%----------------------------

% 1. Confused. I thought that this was basically a proposal for your disseration. Why are your goals written in past-tense? Are they not present or future, you've not completed the work yet, unless I'm unaware.
% 2. You don't need commas before every "and", I've seen this at least twice in this section alone
% 3. Third objective, different levels of yield gains? I'm unclear on what you mean here. Yield gains have a very specific meaning, how are you studying them?

My overall objective is to develop network approaches and apply them to analyze crop health survey data. My first objective for this research is to develop the network model based on crop health survey data and characterize relationships among components of injuries and cropping practices. My second objective is to compare the differential relationships of network models under different seasons or locations. The third objective is to apply network analysis to compare differential interactions in networks under successive (from low to high) levels of estimated actual yields.

% 1. Again, verb tense. Is this work that you've already done? If so, why are you submitting this to the graduate school?
% 2. Grammar in first sentence and second sentence

Once network models based on crop health survey data are constructed, they will be helpful for plant health authorities and the people who related to crop protection especially for rice. The models will support them to design specific strategies for rice pest and disease management and to limit the impacts of these yield reducing factors.  

% eos