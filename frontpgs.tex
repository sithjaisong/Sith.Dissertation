% T I T L E   P A G E
% -------------------
% Last updated May 24, 2011, by Stephen Carr, IST-Client Services
% The title page is counted as page `i' but we need to suppress the
% page number.  We also don't want any headers or footers.
\pagestyle{empty}
\pagenumbering{roman}

% The contents of the title page are specified in the "titlepage"
% environment.
\begin{titlepage}
        \begin{center}
        \vspace*{0.5cm}

        \normalsize
        {\bf NETWORK ANALYSIS OF RICE HEALTH SUREVEY DATA FOR CHARACTERIZATION OF YIELD REDUCING FACTORS AND YIELD LIMITING FACTORS OF TROPICAL RICE ECOSYSTEM IN SOUTH AND SOUTHEAST ASIA}

        \vspace*{4.0cm}

        \normalsize
        \bf{SITH JAISONG} \\

        \vspace*{4.0cm}

        \normalsize
        THESIS OUTLINE SUBMITTED TO THE FACULTY OF GRADUATE SCHOOL\\
        UNIVERSITY OF THE PHILIPPINES LOS BA\~NOS\\ 
 IN FULFILLMENT OF THE \\
        REQUIREMENT FOR THE \\ 
        DEGREE OF \\
        \vspace*{2.0cm}
        PH.D OF SCIENCE \\
        (Plant Pathology) \\

        \vspace*{1.0cm}
%        \copyright\ Pat Neugraad 2007 \\
        May 2015
        \end{center}
\end{titlepage}

% The rest of the front pages should contain no headers and be numbered using Roman numerals starting with `ii'
\pagestyle{plain}
\setcounter{page}{2}

\cleardoublepage % Ends the current page and causes all figures and tables that have so far appeared in the input to be printed.
% In a two-sided printing style, it also makes the next page a right-hand (odd-numbered) page, producing a blank page if necessary.
 
%------------Setting the bibliography style --------------



%------------End of the bibliography style ----------------

% D E C L A R A T I O N   P A G E
% -------------------------------
  % The following is the sample Delaration Page as provided by the GSO
  % December 13th, 2006.  It is designed for an electronic thesis.
%  \noindent
%I hereby declare that I am the sole author of this thesis. This is a true copy of the thesis, including any required final revisions, as accepted by my examiners.

%  \bigskip
  
%  \noindent
%I understand that my thesis may be made electronically available to the public.

%\cleardoublepage
%\newpage

% A B S T R A C T
% ---------------

%\begin{center}\textbf{Abstract}\end{center}

% You're not characterizing different environments. They're all tropical, irrigated, lowland rice
%Characterising rice agroecosytems requires knowledge and information of qualitative and quantitative information. One way of gathering data the data necessary for this is through the conduct of surveys. Given the nature of the data, it is not a simple task to analyse and examine in order to derive information and knowledge. One method in particular, network analysis, has been used to explore the observed relationships between the individual elements in a given system. This study is the first known attempt to apply network analysis to the analysis of in-field and household survey data on rice yield limiting and reducing factors. The data will be analyzed and visualized using the proposed network model. I propose to construct a network model using survey data collected from 2009 to 2011 in several sites (Chekempek, West Java, Indonesia; Mekong River Delta, Vietnam; Tamil Nadu, India; and Suphaburi, Thailand) and test the model with data collected from 2012 to 2015 in several cropping seasons and sites (Chekempek, West Java, Indonesia; Red River Delta, Vietnam; Tamil Nadu and Odisha, India; Suphaburi, Thailand). The anticipated results from this research will be helpful for plant health authorities worldwide, in order to design specific strategies for rice pest and disease management and to limit the impacts of these yield reducing factors. 


%\cleardoublepage
%\newpage

% A C K N O W L E D G E M E N T S
% -------------------------------

%\begin{center}\textbf{Acknowledgements}\end{center}

%I would like to thank all the little people who made this possible.
%\cleardoublepage
%\newpage

% D E D I C A T I O N
% -------------------

%\begin{center}\textbf{Dedication}\end{center}

%\cleardoublepage
%\newpage

% T A B L E   O F   C O N T E N T S
% ---------------------------------
\renewcommand\contentsname{TABLE OF CONTENTS}
\tableofcontents
\cleardoublepage
%\phantomsection
%\newpage

% L I S T   O F   T A B L E S
% ---------------------------
%\addcontentsline{toc}{chapter}{List of Tables}
%\listoftables
%\cleardoublepage
%\phantomsection		% allows hyperref to link to the correct page
%\newpage

% L I S T   O F   F I G U R E S
% -----------------------------
\addcontentsline{toc}{chapter}{LIST OF FIGURES}
\listoffigures
\cleardoublepage
%\phantomsection		% allows hyperref to link to the correct page
%\newpage

% L I S T   O F   S Y M B O L S
% -----------------------------
% To include a Nomenclature section
% \addcontentsline{toc}{chapter}{\textbf{Nomenclature}}
% \renewcommand{\nomname}{Nomenclature}
% \printglossary
% \cleardoublepage
% \phantomsection % allows hyperref to link to the correct page
% \newpage

% Change page numbering back to Arabic numerals
\pagenumbering{arabic}
