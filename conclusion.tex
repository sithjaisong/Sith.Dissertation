%===================
% CONCLUSION
%===================
To manage broad range of pests in sustainable way of agriculture, understandings of yield constrains are required priority. Agroecosystems are diverse and complex. Networks are commonly applied to analyze systemic interplay of biological components. Network analysis enables us to the explore the complex biological process and understand holistically. The gene-gene interaction networks, for instance, reveal the gene functions and system. Besides, the emergence properties after network reconstruction give the clues to cluster the components that close related, called hub. Networks are not static when different environment can reprogram the components arrangements and functions. Moreover, networks can allow us the consider the only the elements response the given changes. With versatile applications of network analysis, agroecosystem potentially can be modeled as a network. The new type of information will be proposed from the network analysis. One of the most challenge will be the model evaluation. Because this is the first attempt to applied network into the context of rice agroecosystem. Meeting this goal will require the development the validated methods to integrate heterogeneous data and built different networks on the basis of the particular rice agroecosystem.